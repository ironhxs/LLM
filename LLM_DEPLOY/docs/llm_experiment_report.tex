\documentclass{llm_experiment_report}
\usepackage{hyperref}

\begin{document}

% 页面风格
\pagestyle{bianyi}
% 封面
\pagestyle{bianyi}

\vspace*{1cm}
\begin{center}
    % 校徽图片
    \makebox[\textwidth][c]{\includegraphics[width=1.2\textwidth]{images/校名.png}}
    
    \vspace{0.8cm}
    
   
    
    \vspace{1.5cm}


    
    {\fontspec{Microsoft YaHei}\textbf{\fontsize{36}{43.2}\selectfont 人工智能期末大作业}}
    
    \vspace{1cm}
    
\end{center}

\vspace{2.5cm}
\begin{table}[h]
	\centering
	\begin{large}
		\begin{tabular}{p{3cm} p{7cm}<{\centering}}
			\sffamily\fontsize{15}{18}\selectfont 指导老师:  & \sffamily\fontsize{15}{18}\selectfont 刘雪南  \\ \cline{2-2}
			\sffamily\fontsize{15}{18}\selectfont 姓\quad\quad 名:  & \sffamily\fontsize{15}{18}\selectfont 贺鑫帅 \\ \cline{2-2}
			\sffamily\fontsize{15}{18}\selectfont 学\quad\quad 号:  & \sffamily\fontsize{15}{18}\selectfont 2023217373 \\ \cline{2-2}
			\sffamily\fontsize{15}{18}\selectfont 班\quad\quad 级:   & \sffamily\fontsize{15}{18}\selectfont 计科23-3 \\ \cline{2-2}
			\sffamily\fontsize{15}{18}\selectfont 提交时间:   & \sffamily\fontsize{15}{18}\selectfont 2025年12月20日 \\ \cline{2-2}
		\end{tabular}
	\end{large}		
\end{table}

\vspace{1.5cm}

\begin{center}
{\Large 计算机与信息学院}
\end{center}

\vspace{0.5cm}

\newpage


% 正文
\section{实验目的和要求}

\subsection{实验目的}

掌握大语言模型的本地化部署方法,理解 RAG(检索增强生成)技术原理,学会使用开源工具构建智能问答系统,提升 AI 应用开发能力。

\subsection{实验要求}

\begin{enumerate}
	\item 完成本地化大模型部署
	\begin{itemize}
		\item 搭建大模型运行环境(GPU / Python / 推理框架等)
		\item 从模型仓库拉取开源大语言模型(Qwen2.5-7B-Instruct)
		\item 完成本地推理服务部署并验证推理能力
	\end{itemize}
	
	\item 调用 API 实现问答功能
	\begin{itemize}
		\item 调用已部署的大模型 API 完成基础问答
		\item 支持对话模式并观察模型响应质量
	\end{itemize}
	
	\item 增强检索式问答(RAG)功能实现
	\begin{itemize}
		\item 选择方案:基于 LangChain 实现 RAG
		\item 构建文本向量化与知识库(使用 FAISS)
		\item 通过 LangChain 集成本地模型与向量检索模块
		\item 使用私有数据集验证增强检索问答效果
	\end{itemize}
\end{enumerate}

\section{实验环境和工具}

\subsection{硬件环境}

\begin{itemize}
	\item GPU:NVIDIA RTX 4060 Laptop(8GB 显存)
	\item CPU:Intel Core i9-14900HX (2.20 GHz)
	\item 内存:32GB DDR4
	\item 存储:954GB SSD
\end{itemize}

\subsection{软件环境}

\begin{itemize}
	\item 操作系统:Windows 11 专业版
	\item Python 版本:3.11.5
	\item CUDA 版本:12.1
	\item 开发工具:VS Code
\end{itemize}

\subsection{核心依赖库}

\begin{table}[H]
	\centering
	\begin{tabular}{lll}
		\toprule
		\textbf{库名称} & \textbf{版本} & \textbf{用途} \\
		\midrule
		torch & 2.1.2+cu121 & PyTorch 深度学习框架 \\
		auto-gptq & 0.6.0 & 4-bit 量化推理加速 \\
		transformers & 4.45.2 & 模型加载与管理 \\
		langchain & 1.1.3 & RAG 框架 \\
		faiss-cpu & 1.13.1 & 向量检索引擎 \\
		flask & 3.1.2 & Web API 服务 \\
		\bottomrule
	\end{tabular}
	\caption{主要依赖库及版本}
\end{table}

\section{实验内容与实现}

\subsection{本地化大模型部署}

\subsubsection{模型选择与下载}

选用 Qwen2.5-7B-Instruct-GPTQ-Int4 量化模型,该模型具备以下特点:

\begin{itemize}
	\item 参数量:70 亿参数
	\item 量化方式:GPTQ 4-bit 量化
	\item 显存需求:约 6GB(适配 RTX 4060 8GB 显存)
	\item 推理速度:15-20 tokens/s
\end{itemize}

模型包含两个分片的 safetensors 文件(\texttt{model-00001-of-00002.safetensors} 和 \texttt{model-00002-of-00002.safetensors})以及配置文件(\texttt{config.json}、\texttt{tokenizer.json} 等)。

\subsubsection{环境配置流程}

关键配置步骤如下:

\begin{enumerate}
	\item 创建 Conda 虚拟环境
\begin{lstlisting}[language=bash]
conda create -n llm_deploy python=3.11
conda activate llm_deploy
\end{lstlisting}
	
	\item 安装 PyTorch(CUDA 12.1 版本)
\begin{lstlisting}[language=bash]
pip install torch==2.1.2+cu121 --index-url 
https://pypi.tuna.tsinghua.edu.cn/simple
\end{lstlisting}
	
	\item 安装 AutoGPTQ 及其他依赖
\begin{lstlisting}[language=bash]
pip install auto-gptq==0.6.0
pip install transformers langchain flask faiss-cpu
\end{lstlisting}
\end{enumerate}

环境配置完成后的依赖列表如图 \ref{fig:env_setup} 所示。

\begin{figure}[htbp]
	\centering
	\includegraphics[width=0.6\textwidth]{images/图1_环境配置.png}
	\caption{环境配置完成}
	\label{fig:env_setup}
\end{figure}

\subsubsection{模型文件准备}

从 ModelScope 下载 Qwen2.5-7B-Instruct-GPTQ-Int4 量化模型,模型文件结构如图 \ref{fig:model_files} 所示。该模型使用 4-bit 量化技术,相比原始 FP16 模型(约 14GB)缩减至 5GB 左右,显存需求降低约 65\%。

\begin{figure}[H]
	\centering
	\includegraphics[width=0.6\textwidth]{images/图2_模型文件.png}
	\caption{模型文件结构}
	\label{fig:model_files}
\end{figure}

\subsubsection{模型加载与验证}

核心模型加载代码如下:

\begin{lstlisting}[language=python]
from auto_gptq import AutoGPTQForCausalLM
from transformers import AutoTokenizer

# 模型路径
model_path = "models/Qwen/Qwen2___5-7B-Instruct-GPTQ-Int4"

# 加载模型
model = AutoGPTQForCausalLM.from_quantized(
    model_path,
    device_map="auto",
    use_cache=True
)

# 加载分词器
tokenizer = AutoTokenizer.from_pretrained(model_path)

print("模型加载成功,显存占用:", torch.cuda.memory_allocated())
\end{lstlisting}

模型加载成功后显存占用约 5.8GB,符合预期。Flask 服务启动日志如图 \ref{fig:api_startup} 所示。

\begin{figure}[H]
	\centering
	\includegraphics[width=0.7\textwidth]{images/图3_API服务启动.png}
	\caption{API 服务启动}
	\label{fig:api_startup}
\end{figure}

\subsection{基础问答功能实现}

\subsubsection{Flask API 服务搭建}

构建 Flask 后端服务,提供 RESTful API 接口:

\begin{lstlisting}[language=python]
from flask import Flask, request, jsonify
from flask_cors import CORS

app = Flask(__name__)
CORS(app)

@app.route('/chat', methods=['POST'])
def chat():
    data = request.json
    user_query = data.get('message', '')
    
    # 构造 Prompt
    prompt = f"User: {user_query}\nAssistant:"
    
    # 模型推理
    inputs = tokenizer(prompt, return_tensors="pt").to("cuda")
    outputs = model.generate(**inputs, max_length=512)
    response = tokenizer.decode(outputs[0], skip_special_tokens=True)
    
    return jsonify({'answer': response})

if __name__ == '__main__':
    app.run(host='0.0.0.0', port=6006)
\end{lstlisting}

服务启动后监听 6006 端口,可通过 Web 界面或 Postman 进行测试。

\subsubsection{基础对话测试}

测试场景包括代码生成、概念解释、数学计算等。图 \ref{fig:basic_chat} 展示了终端对话测试,模型能够正确响应各类问题。

\begin{figure}[H]
	\centering
	\includegraphics[width=0.7\textwidth]{images/图4_终端对话.png}
	\caption{终端对话测试}
	\label{fig:basic_chat}
\end{figure}

\subsection{RAG 增强检索问答实现}

\subsubsection{知识库构建}

准备私有数据集,包含班级档案、课程资料等文档(共约 50KB),文档目录结构:

\begin{lstlisting}[language=bash]
knowledge_base/
├── class_roster_cs23-1.txt   # CS23-1 班级档案
├── class_roster_cs23-2.txt   # CS23-2 班级档案
├── class_roster_cs23-3.txt   # CS23-3 班级档案
├── course_syllabus.md         # 课程大纲
└── course_faq.md              # 常见问题
\end{lstlisting}

文档切分与向量化代码:

\begin{lstlisting}[language=python]
from langchain.text_splitter import RecursiveCharacterTextSplitter
from langchain_community.embeddings import HuggingFaceEmbeddings
from langchain_community.vectorstores import FAISS

# 加载文档
documents = []
for file in os.listdir("knowledge_base"):
    with open(f"knowledge_base/{file}", "r", encoding="utf-8") as f:
        documents.append(f.read())

# 文档切分
text_splitter = RecursiveCharacterTextSplitter(
    chunk_size=500,
    chunk_overlap=50
)
chunks = text_splitter.split_text("\n\n".join(documents))

# Embedding 模型
embeddings = HuggingFaceEmbeddings(
    model_name="models/AI-ModelScope/text2vec-base-chinese"
)

# 构建向量库
vectorstore = FAISS.from_texts(chunks, embeddings)
vectorstore.save_local("vector_store")
\end{lstlisting}

向量库构建完成后,会在 \texttt{vector\_store/} 目录生成 \texttt{index.faiss} 索引文件,共包含 103 个文档片段。知识库文档结构如图 \ref{fig:knowledge_base} 所示。

\begin{figure}[H]
	\centering
	\includegraphics[width=0.7\textwidth]{images/图7_知识库文档.png}
	\caption{知识库文档结构}
	\label{fig:knowledge_base}
\end{figure}

RAG 服务启动日志如图 \ref{fig:rag_startup} 所示,显示向量库加载成功(103 个文档片段)。

\begin{figure}[H]
	\centering
	\includegraphics[width=0.7\textwidth]{images/图6_RAG服务启动.png}
	\caption{RAG 服务启动日志}
	\label{fig:rag_startup}
\end{figure}

\subsubsection{RAG 检索流程实现}

实现 RAG 增强问答的核心逻辑:

\begin{lstlisting}[language=python]
@app.route('/rag_chat', methods=['POST'])
def rag_chat():
    data = request.json
    user_query = data.get('message', '')
    
    # 1. 向量检索(Top-3)
    docs = vectorstore.similarity_search(user_query, k=3)
    context = "\n\n".join([doc.page_content for doc in docs])
    sources = [doc.metadata.get('source', '') for doc in docs]
    
    # 2. 构造增强 Prompt
    prompt = f"""基于以下参考资料回答问题。如果资料中没有相关信息,请明确说明。

参考资料:
{context}

问题:{user_query}

回答:"""
    
    # 3. 模型生成
    inputs = tokenizer(prompt, return_tensors="pt").to("cuda")
    outputs = model.generate(**inputs, max_length=1024)
    answer = tokenizer.decode(outputs[0], skip_special_tokens=True)
    
    return jsonify({
        'answer': answer,
        'sources': sources
    })
\end{lstlisting}

RAG 模式与普通模式的对比效果如图 \ref{fig:rag_comparison} 所示,RAG 能准确回答私有数据集中的问题。

\begin{figure}[H]
	\centering
	\includegraphics[width=0.75\textwidth]{images/图8_RAG模式对比.png}
	\caption{RAG 模式对比}
	\label{fig:rag_comparison}
\end{figure}

相比之下,普通模式(无知识库检索)在面对私有数据问题时无法给出准确答案,如图 \ref{fig:normal_mode} 所示。

\begin{figure}[H]
	\centering
	\includegraphics[width=0.75\textwidth]{images/图9_普通模式回答.png}
	\caption{普通模式回答示例}
	\label{fig:normal_mode}
\end{figure}

此外,模型在代码生成任务上表现优秀,能够生成可运行的快速排序算法,如图 \ref{fig:code_generation} 所示。

\begin{figure}[H]
	\centering
	\includegraphics[width=0.75\textwidth]{images/图10_代码生成测试.png}
	\caption{代码生成测试}
	\label{fig:code_generation}
\end{figure}

\subsection{前端界面实现}

使用 HTML + Tailwind CSS 实现 Glassmorphism 风格聊天界面,支持:

\begin{itemize}
	\item Markdown 渲染(Marked.js)
	\item 代码高亮(Highlight.js)
	\item 信息溯源显示(sources 字段)
\end{itemize}

前端界面采用现代化设计,支持 Markdown 渲染和代码高亮,如图 \ref{fig:frontend} 所示。

\begin{figure}[H]
	\centering
	\includegraphics[width=0.75\textwidth]{images/图5_Web对话界面.png}
	\caption{Web 聊天界面}
	\label{fig:frontend}
\end{figure}

代码块高亮展示效果如图 \ref{fig:code_highlight} 所示,前端使用 Highlight.js 实现语法高亮,提升代码可读性。

\begin{figure}[H]
	\centering
	\includegraphics[width=0.75\textwidth]{images/图11_代码高亮展示.png}
	\caption{代码高亮展示}
	\label{fig:code_highlight}
\end{figure}

\section{实验结果与分析}

\subsection{功能验证}

对比测试普通对话模式与 RAG 模式的差异:

\begin{table}[h]
	\centering
	\begin{tabular}{p{5cm}p{4cm}p{4cm}}
		\toprule
		\textbf{测试问题} & \textbf{普通模式} & \textbf{RAG 模式} \\
		\midrule
		计科23-3班的辅导员是谁? & "我不知道" 或编造答案 & "玄奘大师"(准确) \\
		计科23-3班班长是谁? & 无法回答 & "孙悟空"(准确) \\
		计科2班班长是谁? & 无法回答 & "关羽"(准确) \\
		\bottomrule
	\end{tabular}
	\caption{对比测试结果}
\end{table}

从测试结果可以看出,RAG 模式能够准确回答私有数据集中的问题,而普通模式只能依赖预训练知识,无法获取本地文档中的信息。如图 \ref{fig:rag_comparison} 所示,RAG 检索成功返回了 3 个相关文档来源(\texttt{cs\_23\_3.txt}、\texttt{cs\_23\_1.txt}、\texttt{cs\_23\_2.txt}),并基于这些文档给出了准确答案。

\subsection{性能分析}

\begin{itemize}
	\item \textbf{响应时间}:平均 2 秒(含检索 + 推理)
	\item \textbf{检索准确率}:Top-3 召回率约 85\%
	\item \textbf{显存占用}:稳定在 6GB 左右
	\item \textbf{推理速度}:15-20 tokens/s
\end{itemize}

\subsection{关键问题与解决}

\subsubsection{AutoGPTQ 版本兼容性}

\textbf{问题}:安装 AutoGPTQ 后报错 \texttt{Couldn't find CUDA library}

\textbf{解决}:必须严格匹配 \texttt{torch==2.1.2+cu121} 版本,并从清华镜像源安装 CUDA 版本的 PyTorch。

\textbf{参考资料}:\url{https://jishuzhan.net/article/1868560413910634498}

\subsubsection{FAISS 序列化安全警告}

\textbf{问题}:加载向量库时报 \texttt{dangerous deserialization} 错误

\textbf{解决}:在 \texttt{FAISS.load\_local()} 时添加参数 \texttt{allow\_dangerous\_deserialization=True}。

\section{实验总结}

\subsection{主要收获}

\begin{enumerate}
	\item \textbf{技术能力提升}
	\begin{itemize}
		\item 掌握了大模型量化部署技术(GPTQ 4-bit)
		\item 学会了 RAG 架构的完整实现流程
		\item 熟悉了向量检索技术(FAISS)的应用
	\end{itemize}
	
	\item \textbf{工程能力锻炼}
	\begin{itemize}
		\item 学会了环境依赖管理和版本锁定
		\item 提升了系统性排查问题的能力
		\item 积累了 AI 应用全栈开发经验
	\end{itemize}
	
	\item \textbf{技术认知深化}
	\begin{itemize}
		\item 理解了 RAG 如何解决大模型"幻觉"问题
		\item 认识到私有数据集对企业应用的重要性
		\item 体会到开源生态在 AI 民主化中的价值
	\end{itemize}
\end{enumerate}

\subsection{不足与改进}

\begin{itemize}
	\item \textbf{并发处理}:当前 Flask 单线程架构不支持并发,可改用 FastAPI 或 Gunicorn 多进程
	\item \textbf{检索优化}:可引入 Reranking 机制提升复杂查询的召回率
	\item \textbf{知识库管理}:实现增量更新机制,避免每次修改都需全量重建
	\item \textbf{多轮对话}:增加会话管理,支持上下文保持
\end{itemize}

\subsection{心得体会}

通过本次实验,我深刻体会到 AI 应用开发与传统软件开发的差异。大模型的不确定性、环境配置的复杂性、以及 RAG 系统的调优难度,都对开发者提出了更高要求。但正是这些挑战让我意识到:AI 技术的落地不仅需要理论知识,更需要扎实的工程能力和持续的学习精神。

开源生态的力量令人惊叹——从 Qwen 模型到 LangChain 框架,再到 FAISS 向量库,这些工具让个人开发者也能构建专业级的 AI 应用。未来,随着模型能力的持续提升和工具链的不断完善,基于 RAG 的私有知识助手必将成为企业和个人的标配工具。

\subsection{项目代码}

本实验完整代码已开源至 GitHub:\url{https://github.com/ironhxs/LLM}

\end{document}
